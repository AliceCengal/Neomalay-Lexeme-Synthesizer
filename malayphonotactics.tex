\documentclass{article}
\usepackage{booktabs}
\usepackage[T1]{fontenc}
\usepackage{siunitx}
\usepackage{textcomp}
\setlength{\parindent}{0em}
\setlength{\parskip}{1em}
\newcommand{\grapheme}[1]{\textlangle#1\textrangle}
\begin{document}

\begin{center} {\LARGE NEO-MALAY LEXEME SYNTHESIZER}\\ \bigskip 
Athran Abdul Rahman\\
Universiti Teknologi Malaysia\\
September 2017\end{center}

\textbf{Neo-malay} here means that the language to be used in this study is not exactly any extant Malay language variety, but rather is a constructed language inspired by the Malay language.

\textbf{Lexeme Synthesizer} is a procedure whereby lexemes are synthesized according to the phonological and phonotactical rules of a language.

\textbf{Goal} --- to synthesize new lexemes that visually and aurally resemble extant native Malay words.

\textbf{Significance} --- a lexeme synthesizer may be useful for building a \textit{lorem ipsum} generator localized for the Malay language. It may also be useful for inventing new names for writing fictions.

\textbf{Methods}
\begin{enumerate}
    \item Analyse the phonology of Malay, then derive a phonology for Neomalay
    \item Analyse the phonotactic rule of Malay, then derive a phonotactic rule for Neomalay
    \item Write a computer program that uses the rules set out in Step 1 \& 2 to generate new lexemes in Neomalay
\end{enumerate}

\section{Phonology}

The phonology of the Malay language is analysed, and a phonology is devised for the Neomalay language. In this paper, the character \grapheme{\~y} is used instead of \grapheme{ny}, and \grapheme{\~g} instead of \grapheme{ng}. The symbols \grapheme{j}, \grapheme{c} and \grapheme{y} are used according to the Malay orthographic convention instead of the IPA standard.

\subsection{Consonants}

The consonant register seems to be mostly consistent across the Standard Malay languages in the Nusantara. Most analysis recognise the distinction between the native malay phonemes and the foreign phonemes. For this synthesis, only the native consonant phonemes are used, consisting of 18 consonants. The following is one analysis of the Malay consonant register by Adelaar.

\begin{center}
    \begin{tabular}{l c c c c c}
        \toprule
                       & labial & dental & alveolar & palatal & velar \\ \midrule
        Voiceless stop & p      & t      &          & c       & k     \\
        Voiced stop    & b      &        & d        & j       & g     \\
        Nasal          & m      &        & n        & \~y     & \~g   \\
        Semivowel      & w      &        &          & y       &       \\ \midrule
        Others         &        \multicolumn{5}{c}{s, h, l, r}        \\ \bottomrule
    \end{tabular}
\end{center}

%Soderberg and Olsen

Clynes and Deterding presented an alternate consonant patterning based on the active articulator, instead of the passive articulator as it is usually done. 

\begin{center}
    \begin{tabular}{l c c c c}
        \toprule
                  & Labial & Apical & Laminal & Dorsal \\ \midrule
        Plosive   & p b    & t d    & c j     & g k    \\
        Fricative &        &        & s       & h      \\
        Nasal     & m      & n      & \~y     & \~g    \\
        Others    & w      & r l    & y       &        \\ \bottomrule
    \end{tabular}
\end{center}

\subsection{Vowels}

Standard Malay is generally described as having six vowel: front-close, front-mid, central-mid, central-open, back-close and back-mid. However, there is a situation of degeneracy between the close and mid varieties of the front and back vowels in the final syllable. Additionally, Bruneian Malay is described as having only three vowels \{a, i, u\}. The following is a simplified vowel register devised for Neomalay, consisting only of four vowels.

\begin{center}
    \begin{tabular}{l c c c}
        \toprule
              & Front & Central & Back \\ \midrule
        Close & i     &         & u    \\
        Mid   &       & e       &      \\
        Open  &       & a       &      \\ \bottomrule
    \end{tabular}
\end{center} \bigskip

The merging of the close-mid vowel pairs aims to simplify the synthesis process. The phonetic distinction can be regenerated afterwards according to their own rules.

\pagebreak
\section{Phonotactics}

\begin{quote}
    ``More than 90\% of the native lexicon is based on disyllabic root morphemes, with small percentages of monosyllabic and trisyllabic roots.'' (Adelaar 1992)
\end{quote}

The root lexeme prototype to be used in this synthesis is given below, consisting of two syllables, each having the C1--V--C2 structure.

\begin{center}
    C1 -- V1 -- C2 -- C3 -- V2 -- C4
\end{center}

All but one of the consonants are optional, with the minimal lexeme having the structure V1--C3--V2: ara, apa, api, ura, ubi, itu, etc. The following rules are adapted from Adelaar and Clynes \& Deterding.

\textbf{C1 \& C3}, syllable onset -- Any consonant

\textbf{C1 \& C3 homorganic rule} -- When both C1 and C3 are plosives, they cannot have the same articulation but different voicing. Thus, /b-b-/, /b-k-/ and /b-g-/ are allowed but not /b-p-/ and /g-k-/.

\textbf{C2 \& C4}, syllable coda -- Can't be laminals \{c, j, \~y\} and voiced plosives \{b, d, g\}

\textbf{C2} -- Also can't be voiceless plosives and /h/. If C2 is /s/, then C3 must be a voiceless plosive: puspa, pasti, pasca, laskar.

\textbf{C2 nasal \& C3 homorganic rule} -- When C2 is a nasal, it must have the same articulation with C3, and C3 can only be plosives and /s/, but not \~y. Thus, /-nd-/ is allowed, but not /-mg-/. For the laminals, /c/ and /j/ are preceded by /n/ while /s/ is preceded by /\~g/.

\textbf{C2 /r/ and \& C3} -- When C2 is /r/, C3 can't be /h/, /w/ and /j/.

\textbf{No gemination} -- C2 and C3 must not be the same consonant.

\subsection*{V1 \& V2 -- Vowels}

V1 and V2 can be monopthongs only. V1 can be any vowel, while V2 can't be /e/.

The syllable's vowel puts a restriction on the possible semivowels that can occupy the coda. Standard Malay has been described as having only monophtongs, and what seems like diphtongs (ked\textbf{ai}) can instead be analysed as a vowel-semivowel sequence (k-e-d-a-y). This is supported by the fact that no consonant ever follow the phonemes /ai/, /au/ and /ui/ in the same syllable, thus it can be argued that the semivowel occupies the single consonant coda position.

\begin{center}
    \begin{tabular}{c c c}
        \toprule
        Semivowel & y  & w  \\ \midrule
            a     & ay & aw \\
            e     & -  & -  \\
            i     & -  & iw \\
            u     & uy & -  \\ \bottomrule
    \end{tabular}
\end{center}

A semivowel cannot follow /e/, nor precede it. The vowel-semivowel mostly occurs in the root-final position (V2--C4). The rare instances where it occurs in the root-initial position (V1--C2) are loanwords: `haiwan', `kailan'. For this synthesis, I decided to prohibit root-initial vowel-semivowel. /iw/ seems to be very rare, but some examples include the place name `Setiu' and the folk hero `Hang Lekiu'.

Some other sequences that look like diphtongs are analysed as V-SV-V sequences.

\begin{center}
    \begin{tabular}{l c c c c c c}
        \toprule
        Word  & \hspace{2em} & \multicolumn{5}{c}{Analysis} \\
              &              &   & V & SV & V &             \\ \midrule
        kuat  &              & k & u & w  & a & t           \\
        buah  &              & b & u & w  & a & h           \\
        liar  &              & l & i & y  & a & r           \\
        laung &              & l & a & w  & u & \~g         \\
        air   &              &   & a & y  & i & r           \\ \bottomrule
    \end{tabular}
\end{center}

This implies that V1 also restricts the semivowel in C3 if C2 is absent. Put in another way, if C3 is a semivowel, then C2 must be null. %If V1 is /e/, then one of C1 or C2 must not be null. 

That said, there are the /ian/ and /uan/ sequences, found in `kalian', `haruan', `durian', `tebuan'. The literature that I've found so far made no mention of these sequences, which appear to be legitimate diphtongs, although they could be analysed as trisyllabic roots as well, or disyllabic roots with a fossilized suffix. They will be ignored for now and are assumed to be forbidden.

%Whatever there is to it will not be.

%Assimilation of me\~g-
%
%- Dropped before sonorants: nasals \{m, n, \~j, \~g\}, liquids \{l,r\}, semivowels \{w, y\}
%
%- Assimilated before obstruents \bigskip
%
%\begin{center}
%    \begin{tabular}{l c c c c c}
%        \toprule
%        Voiceless & -mp*- & -nt*- & -nc-         & -\~gk*- &         \\
%        Voiced    & -mb-  & -nd-  & -nj-         & -\~gg-  & -\~g'*- \\
%        Fricative &       &       & -\~js*-/-ns- &         & -\~gh*- \\ \bottomrule
%    \end{tabular}
%\end{center}
%
%* Voiceless components (except for /c/) may be dropped \bigskip

\section{Phonetic transformation}

A number of transformation is applied to the generated lexeme in order to make it more visually and aurally similar to the vernacular Malay language.

\textbf{Semivowel grapheme} -- The semivowels at the coda position are written as /i/ and /u/. Additionally, /uy/ is converted to /oi/.

\textbf{Vowel glide} -- For the case of V--SV--V, the middle semivowel is elided and the two vowels are pronounced as a glide. This is an arbitrary choice as different dialects pronounce this sequence differently.

\textbf{Lowering the front and back vowels} -- In the second syllable, for closed syllables, the vowels /i/ and /u/ are lowered to /e/ and /o/. Vowel harmonisation also occurs. If the first and second vowels are the same, then the first vowels are also lowered.

\section{Lexeme formation}

The Neomalay language described above has 18 consonants (19 if counting null) and 4 vowels. Before the phonotactic rules are considered, the number of lexemes that can be generated using the CVCCVC prototype is

\[ 19 \times 4 \times 19 \times 18 \times 4 \times 19 = \num{1975392} \]

After the application of all phonotactic rules that I have so far, the number of lexemes generated is \num{81226}. Below is a random sampling of the final result of the synthesizer.

\begin{center}
\begin{tabular}{c c c c c c c c}
    \toprule
    raskeng & ngumem  & ngoyos  & nyesper & ngangkah & rocop    & rongkot & iscam   \\
     wecel  & menjam  & langgu  & lospol  & mabor    & kosong   & hances  & muku    \\
    nguyau  & kescet  & rimbas  & teskau  & mescau   & teskek   & dapi    & nyewes  \\
    descan  & ecoi    & kidom   & senger  & reteh    & jimak    & jisku   & monggoh \\
     raget  & huscu   & lujel   & anyom   & lemes    & hutek    & jegu    & recoh   \\
     musci  & nyejong & betem   & wantap  & loscok   & nagan    & rapos   & temen   \\
    ngangah & sesti   & hicor   & isoi    & jinyak   & askas    & mesop   & penggel \\
     serem  & mecep   & hingsar & uspang  & pesel    & deceng   & udep    & ebol    \\
    mespel  & pangses & lestel  & gestep  & nunges   & cegan    & becoi   & gemem   \\
     buhas  & nyungau & tispan  & nyagang & dohor    & watong   & heskos  & beler   \\
    buncang & lesi    & mendep  & mahek   & bagi     & pinggang & guban   & kescang \\ \bottomrule
\end{tabular}
\end{center}

\end{document}